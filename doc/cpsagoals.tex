\documentclass[12pt]{article}
\usepackage{amssymb}
\usepackage{amsmath}
\usepackage{url}
\usepackage{graphicx}
\usepackage{makeidx}
\usepackage{alltt}
\input{macros}

\makeindex

\title{CPSA and Formal Security Goals}
\author{John D.~Ramsdell\\
  The MITRE Corporation\\ CPSA Version \version}

\begin{document}
\maketitle
\cpsacopying

\tableofcontents

\listoffigures

\listoftables

\newpage

\nocite{Ramsdell12,GuttmanLiskovRowe14,Guttman14}

\section{Introduction}

\begin{sloppypar}
Analyzing a cryptographic protocol means finding out what security
properties---essentially, authentication and secrecy properties---are
true in all its possible executions.
\end{sloppypar}

Given a protocol definition and some assumptions about executions,
{\cpsa} attempts to produce descriptions of all possible executions of
the protocol compatible with the assumptions.  Naturally, there are
infinitely many possible executions of a useful protocol, since
different participants can run it with varying parameters, and the
participants can run it repeatedly.

However, for many naturally occurring protocols, there are only
finitely many of these runs that are essentially different.  Indeed,
there are frequently very few, often just one or two, even in cases
where the protocol is flawed.  We call these essentially different
executions the \emph{shapes} of the protocol.  Authentication and
secrecy properties are easy to ``read off'' from the shapes, as are
attacks and anomalies, according to the introduction in the {\cpsa}
Primer~\cite{cpsaprimer09}.

But how easy is it to read off authentication and secrecy properties?
What precisely is it that an expert sees?  This paper describes
{\cpsa}'s support for reasoning about security goals using first-order
logic.  With security goals expressed in first-order logic, intuition
is replaced by model-finding and theorem-proving.

\section{Overview}

\emph{Description of workflow.}

\bibliography{cpsa}
\bibliographystyle{plain}

\printindex

\end{document}
