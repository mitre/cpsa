\documentclass[landscape]{slides}
\usepackage{mitreslides}
\usepackage{amssymb}
\usepackage{amsmath}
\usepackage{color}
\usepackage{graphicx}
\usepackage{alltt}
\usepackage{url}
%\usepackage{hyperref}

\newcommand{\cpsa}{CPSA}
\DeclareMathOperator{\domm}{\mathit{dom}}
\newcommand{\cn}[1]{\ensuremath{\mathop{\relax
                    \smash{\sf#1}}\!\mathop{\vphantom{#1}}\nolimits}}
\newcommand{\dom}[1]{\ensuremath{\mathop{\relax
                    \smash{\rm#1}}\!\mathop{\vphantom{#1}}\nolimits}}
\newcommand{\fn}[1]{\ensuremath{\mathop{\relax
                    \smash{\it#1}}\!\mathop{\vphantom{#1}}\nolimits}}
\newcommand{\termalg}{\ensuremath{\mathfrak T}}
\newcommand{\mesgalg}{\ensuremath{\mathfrak A}}
\newcommand{\tr}{\ensuremath{\mathfrak C}}
\newcommand{\ints}{\ensuremath{\mathbb Z}}
\newcommand{\bundle}{\ensuremath{\mathcal{B}}}
\newcommand{\nterm}[1]{\ensuremath{\langle\mathit{#1}\rangle}}
\newcommand{\nterms}[1]{\ensuremath{\nterm{#1}^\ast}}
\newcommand{\ntermp}[1]{\ensuremath{\nterm{#1}^+}}
\newcommand{\ntermo}[1]{#1$^?$}
\newcommand{\enc}[2]{\{\!|#1|\!\}_{#2}}
\newcommand{\invk}[1]{{#1}^{-1}}
\newcommand{\pubk}[1]{K_{#1}}
\newcommand{\privk}[1]{K^{-1}_{#1}}
\newcommand{\eqq}{\stackrel{?}{=}}
\newcommand{\hash}[1]{\#(#1)}
\newcommand{\inbnd}{\mathord -}
\newcommand{\outbnd}{\mathord +}
\newcommand{\cert}{\mathsf{cert}}
\newcommand{\valid}{\mathsf{valid}}
\newcommand{\data}{\mathsf{data}}
\newcommand{\says}{\mathbin{\mathrm{says}}}
\DeclareMathOperator{\ok}{\mathit{ok}}
\DeclareMathOperator{\id}{\mathit{id}}
\DeclareMathOperator{\resource}{\mathit{resource}}
\DeclareMathOperator{\approved}{\mathit{approved}}
\DeclareMathOperator{\ask}{\mathit{ask}}
\DeclareMathOperator{\meas}{\mathit{meas}}
\DeclareMathOperator{\verifier}{\mathit{verifier}}
\DeclareMathOperator{\auth}{\mathit{auth}}
\DeclareMathOperator{\epca}{\mathit{epca}}
\DeclareMathOperator{\ltk}{\mathsf{ltk}}
\newcommand{\ctosa}{\enc{R,A,K}{K_S}}
\newcommand{\stova}{\enc{S,A,R,N_S}{K_V}}
\newcommand{\ptova}{\enc{\cert,A,I,E}{\invk{K}_E}}
\newcommand{\vtosa}{\enc{N_V,J,V,N_S,M}{K_S}}
\newcommand{\stoca}{\enc{N_V,J,V,M}{K}}
\newcommand{\ctoaa}{\enc{S, N_V,J,V,M,R}{\ltk(A,A)}}
\newcommand{\atoca}{\enc{K',N_V,B}{\ltk(A,A)}}
\newcommand{\blob}{\enc{K',S,J_O,M,P,
\enc{\hash{\hash{A,V,N_V,J_O},M,P}}{\invk{}}}{K_V}}
\newcommand{\vtosb}{\enc{\valid,N_S,K'}{K_S}}
\newcommand{\stocb}{\enc{\data,D}{K'}}
\newcommand{\seq}[1]{\langle#1\rangle}
\newcommand{\append}{\mathbin{{}^\frown}}
\newcommand{\flow}[3]{\ensuremath{#1,#2\rhd#3}}
\newcommand{\infer}[2]{\begin{array}{c}#1\\\hline#2\end{array}}
\newcommand{\inferb}[2]{\begin{array}[b]{c}#1\\\hline#2\end{array}}
\newcommand{\infertwo}[3]{\inferb{#1\quad#2}{#3}}
\newcommand{\termat}{\mathbin{@}}
\newcommand{\all}[1]{\forall#1\mathpunct{.}}
\newcommand{\some}[1]{\exists#1\mathpunct{.}}
\newcommand{\idsigma}{\sigma_{\mathrm{id}}}

\newenvironment{zitemize}
{\begin{itemize}
\setlength{\itemsep}{0em}
\setlength{\topsep}{0em}
\setlength{\partopsep}{0em}
\setlength{\parsep}{0em}
\setlength{\parskip}{0em}}%
{\end{itemize}}

\begin{document}

\begin{titleslide}{Symbolic Cryptographic Protocol \\ Analysis using CPSA}
%
John D.~Ramsdell\\
~\\
The MITRE Corporation\\
~\\
October 2012
\end{titleslide}

\begin{mitreslide}{Cyptographic Protocol Shapes Analyzer}
\begin{description}
\item [Goal:] Protocol analysis using Dolev-Yao model
\item [Foundation:] Strand Spaces
\item [Distribution:] \url{https://github.com/mitre/cpsaexp}
\item [Contributors:]
  Joshua D. Guttman, John D. Ramsdell, Jon C. Herzog, Shaddin
  F. Doghmi, F. Javier Thayer, Paul D. Rowe, and Moses D. Liskov
\end{description}
\end{mitreslide}

\begin{note}
  Pages of this form attempt to fill in the spoken part of a
  class using these slides.

  The Cryptographic Protocol Shapes Analyzer ({\cpsa}) attempts to
  enumerate all essentially different executions possible for a
  cryptographic protocol. We call them the \emph{shapes} of the
  protocol. Naturally occurring protocols have only finitely many,
  indeed very few shapes. Authentication and secrecy properties are
  easy to determine from them, as are attacks and anomalies.

  This presentation gives a very brief tour of the tool and how it is
  used in practice.
\end{note}

\begin{mitreslide}{Anatomy of a CPSA Installation}
\begin{zitemize}
\item Programs
\begin{zitemize}
\item \texttt{cpsa4}: main analysis tool
\item \texttt{cpsa4shapes}: extracts shapes from \texttt{cpsa} output
\item \texttt{cpsa4graph}: visualize output using XHTML and SVG
\item \texttt{cpsa4diff}: compare two \texttt{cpsa} output files
\item \texttt{cpsa4sas}: produce shape analysis sentences
\item \texttt{cpsa4pp}: pretty print input and output
\item Build tools: GNU makefile template and a Haskell script
\end{zitemize}
\item Documentation
\begin{zitemize}
\item User documentation: user guide, primer, and goal description
\item Sample input: Needham-Schroeder, Blanchet, Woo-Lam, Otway-Rees,
  Yahalom, ffgg, goals, rules
\end{zitemize}
\end{zitemize}
\end{mitreslide}

\begin{note}
  Although there are many programs in the {\cpsa} package, typical
  usage is very simple.
\end{note}

\begin{mitreslide}{Using CPSA}
\begin{itemize}
\item With a text editor, user enters a description of a
\begin{zitemize}
\item protocol as a set of roles
\item problem point-of-view\\ (what is assumed to have happened)
\end{zitemize}
\item User runs the tool (\texttt{\$ echo build | ghci Make4.hs} \\
  or on Windows, click on \texttt{Make4.hs} and type \texttt{build})
\item In a web browser, user views the output that shows
\begin{zitemize}
\item what else CPSA inferred must have happened, or
\item all CPSA steps used to produce the answers
\end{zitemize}
\end{itemize}
\end{mitreslide}

\begin{note}
  At this point, if you have not already done so, follow the
  instructions in \url{index.html} and analyze the examples.  The rest
  of these slides will focus on Blanchet's ``Simple Example
  Protocol'', so open \url{blanchet.xhtml}.  Click on
  Derivation Tree 9.
\end{note}

\begin{mitreslide}{Blanchet's ``Simple Example Protocol''}
$$\begin{array}{r@{{}\colon{}}l}
A\to B&\enc{\enc{S}{\privk{A}}}{\pubk{B}}\\
B\to A&\enc{D}{S}
\end{array}$$
\begin{center}
CPSA Style Roles \\[2ex]
\begin{tabular}{c}
Initiator ($\fn{init}$ role)\\[3ex]
\includegraphics{blanchet-0.mps}
\end{tabular}\hfil
\begin{tabular}{c}
Responder ($\fn{resp}$ role)\\[3ex]
\includegraphics{blanchet-1.mps}
\end{tabular}\\[3ex]
\end{center}
\end{mitreslide}

\begin{note}
  Blanchet's protocol has two steps.
  Alice sends to Bob a freshly generated symmetric key~$S$ signed with Alice's
  private key~$\privk{A}$ and then encrypted with Bob's public
  key~$\pubk{B}$.  Bob sends data~$D$ encrypted with the symmetric key.

  Locate the description of the protocol in \url{blanchet.xhtml} in
  the third derivation tree.
\end{note}

\begin{mitreslide}{Blanchet Protocol in CPSA}
\begin{quote}
\begin{verbatim}
(defprotocol blanchet basic
  (defrole init
    (vars (a b name) (s skey) (d data))
    (trace
     (send (enc (enc s (privk a)) (pubk b)))
     (recv (enc d s))))
  (defrole resp
    (vars (a b name) (s skey) (d data))
    (trace
     (recv (enc (enc s (privk a)) (pubk b)))
     (send (enc d s)))))
\end{verbatim}
\end{quote}
\end{mitreslide}

\begin{note}
  Locate the description of scenario in \url{blanchet.xhtml} in
  Derivation Tree 9.
\end{note}

\begin{mitreslide}{Blanchet Problem Statement}
\begin{itemize}
\item Analyze from the point-of-view of a full length responder
\item Assume $\privk{A}$ and $\privk{B}$ are uncompromised
\item Assume symmetric key~$S$ is freshly generated
\hfill\raisebox{0ex}[0ex]{\includegraphics{blanchet-5.mps}}
\end{itemize}
\begin{quote}
\begin{verbatim}
(defskeleton blanchet
  (vars (a b name) (s skey) (d data))
  (defstrand resp 2 (a a) (b b) (s s) (d d))
  (non-orig (privk a) (privk b))
  (uniq-orig s))
\end{verbatim}
\end{quote}
\end{mitreslide}

\begin{note}
  {\cpsa} takes one step to find the shape associated with the scenario.
  Locate the shape for the scenario in this derivation tree.
\end{note}

\begin{mitreslide}{Blanchet Shape}
\begin{quote}
\begin{alltt}
(defskeleton blanchet
  (vars (d data) (a b \textcolor{blue}{b-0} name) (s skey))
  (defstrand resp 2 (d d) (a a) (b b) (s s))
  (defstrand init 1 (a a) (b \textcolor{blue}{b-0}) (s s))
  (precedes ((1 0) (0 0)))
  (non-orig (privk a) (privk b))
  (uniq-orig s)
  ...)
\end{alltt}
\end{quote}
\end{mitreslide}

\begin{note}
  Observe that the responder role variable \texttt{b} maps to skeleton
  variable \texttt{b}, but the initiator role variable \texttt{b} maps
  to skeleton variable \texttt{b\_0}.  The two strands disagree on
  Bob's identity when agreement is expected.  The dotted line in
  Skeleton 10 indicates the message transmitted differs from the one sent.

  In the next slide, the traces show the implication of the
  disagreement on the instantiations of variable \texttt{b}.
\end{note}

\begin{mitreslide}{Blanchet Shape with Traces}
\begin{quote}
\begin{alltt}
(defskeleton blanchet
  (vars (d data) (a b \textcolor{blue}{b-0} name) (s skey))
  (defstrand resp 2 (d d) (a a) (b b) (s s))
  (defstrand init 1 (a a) (b \textcolor{blue}{b-0}) (s s))
  ...
  (traces
    ((recv (enc (enc s (privk a)) (pubk b)))
     (send (enc d s)))
    ((send (enc (enc s (privk a)) (pubk \textcolor{blue}{b-0})))))
  ...)
\end{alltt}
\end{quote}
\end{mitreslide}

\begin{mitreslide}{Blanchet Shape}
\begin{center}
\includegraphics{blanchet-4.mps}
\end{center}
\begin{quote}
\begin{alltt}
(defskeleton blanchet
  (vars (d data) (a b \textcolor{blue}{b-0} name) (s skey))
  (defstrand resp 2 (d d) (a a) (b b) (s s))
  (defstrand init 1 (a a) (b \textcolor{blue}{b-0}) (s s))
  (precedes ((1 0) (0 0)))
  ...)
\end{alltt}
\end{quote}
\end{mitreslide}

\begin{note}
  The operation field describes the authentication test that was
  solved to produce the shape.  Authentication tests are introduced in
  \url{cpsaprimer.pdf}.
\end{note}

\begin{mitreslide}{Blanchet Shape with Operation}
\begin{quote}
\begin{alltt}
(defskeleton blanchet
  (vars (d data) (a b \textcolor{blue}{b-0} name) (s skey))
  (defstrand resp 2 (d d) (a a) (b b) (s s))
  (defstrand init 1 (a a) (b \textcolor{blue}{b-0}) (s s))
  ...
  (operation
    encryption-test       ; Authentication test type
    (added-strand init 1) ; Regular augmentation
    (enc s (privk a))     ; Critical message
    (0 0))                ; Test node
  ...)
\end{alltt}
\end{quote}
\end{mitreslide}

\begin{note}
  The operation field explains why {\cpsa} produced strands that do
  not agree on the identity of \texttt{b}.  The critical message does
  not refer to it, so there is nothing that forces agreement.

  Navigate to Derivation Tree 23 in \url{blanchet.xhtml}.
\end{note}

\begin{mitreslide}{Corrected Blanchet Example Protocol}
$$\begin{array}{r@{{}\colon{}}l}
A\to B&\enc{\enc{S,B}{\privk{A}}}{\pubk{B}}\\
B\to A&\enc{D}{S}
\end{array}$$
\begin{center}
CPSA Style Roles \\[2ex]
\begin{tabular}{c}
Initiator ($\fn{init}$ role)\\[2ex]
\includegraphics{blanchet-2.mps}
\end{tabular}\hfil
\begin{tabular}{c}
Responder ($\fn{resp}$ role)\\[2ex]
\includegraphics{blanchet-3.mps}
\end{tabular}\\[3ex]
\end{center}
\end{mitreslide}

\begin{mitreslide}{Corrected Blanchet Protocol in CPSA}
\begin{quote}
\begin{alltt}
(defprotocol blanchet-fixed basic
  (defrole init
    (vars (a b name) (s skey) (d data))
    (trace
     (send (enc (enc s \textcolor{blue}{b} (privk a)) (pubk b)))
     (recv (enc d s))))
  (defrole resp
    (vars (a b name) (s skey) (d data))
    (trace
     (recv (enc (enc s \textcolor{blue}{b} (privk a)) (pubk b)))
     (send (enc d s)))))
\end{alltt}
\end{quote}
\end{mitreslide}

\begin{note}
  The corrected protocol ensures the critical message refers to
  \texttt{b}.  As a result, {\cpsa} infers the desired agreement.
\end{note}

\begin{mitreslide}{Corrected Blanchet Shape}
\begin{quote}
\begin{alltt}
(defskeleton blanchet-fixed
  (vars (d data) (a b name) (s skey))
  (defstrand resp 2 (d d) (a a) (b b) (s s))
  (defstrand init 1 (a a) (b b) (s s))
  (precedes ((1 0) (0 0)))
  (non-orig (privk a) (privk b))
  (uniq-orig s)
  (operation encryption-test
    (added-strand init 1)
    (enc s \textcolor{blue}{b} (privk a))   ; Critical message
    (0 0))
  ...)
\end{alltt}
\end{quote}
\end{mitreslide}

\end{document}
