\chapter{Setup and Installation}
\label{ch:setup}

\section{Basic Installation}

The use of {\cpsa} requires Haskell, the programming language in which
{\cpsa} is coded.  Our recommendation is to use the Haskell Platform,
which is available for Windows, Mac, and Linux.  The ``Core'' platform
is sufficient, but the ``Full'' platform is also fine.  On Linux,
install the \texttt{haskell-platform} package.  Otherwise, follow the
download and installation instructions at:

\begin{center}
\url{http://www.haskell.org/platform/}
\end{center}

To install {\cpsa}, run in a terminal or command prompt:

\noindent \texttt{\$ cabal update} (to get the latest package list; this may take a while.)\\
\texttt{\$ cabal install cpsa}

Note that if you are behind a proxy, you may have to set the http
proxy for your shell if you haven't already. For example, on a mac:\\
\texttt{\$ export http\_proxy=http://proxy.myorg:port}\\
On Windows:\\
\texttt{\$ set HTTP\_PROXY=http://proxy.myorg:port}

Cabal will install {\cpsa} in a directory specified in its config file
(usually in ~/.cabal/config, unless you've installed Haskell in a
different directory). Instructions for changing your path are included
in the config file. The final step in the cabal install process should
print the location that {\cpsa} has been installed in.

\subsection{Getting the Source}
If you have trouble with \texttt{cabal}, or if you'd like extra
features such as the {\cpsa} test suite of example protocols, you can
download the current source distribution directly~at:

\begin{center}
  \url{http://github.com/mitre/cpsa}
\end{center}

There is a directory named \texttt{cpsa4} at the top-level of the
repository.  It contains a copy of the {\cpsa} sources downloaded and
compiled using \texttt{cabal}.  On all platforms, to install from this
source, change into the \texttt{cpsa4} directory and type:

\begin{verbatim}
$ cabal update
$ cabal install parallel
$ cabal configure
$ cabal build
$ cabal install
\end{verbatim}

Alternatively, there are other install options described in
\texttt{README.txt} in the \texttt{cpsa4} directory.

\section{Finding Documentation}
{\cpsa} comes with documentation, but it can be difficult to locate by
hand when the tool has been installed through cabal.  Run

\texttt{\$ cpsa4 -h}

\noindent
to see the program's help message, including the documentation
directory, where this manual should be found.

\section{Running {\cpsa}}
\label{sec:running}

To analyze a protocol you have put in prob.scm type:
\begin{verbatim}
$ cpsa4 -o prob.txt prob.scm
$ cpsa4shapes -o prob_shapes.txt prob.txt
$ cpsa4graph -o prob_shapes.xhtml prob_shapes.txt
\end{verbatim}

See Section~\ref{sec:options} for command-line options.

The \texttt{cpsa4shapes} command is optional, but recommended; it cuts
down {\cpsa}'s output to only show final results. Unless you're
doing detailed debugging, using it will make the output much easier to
read.

To analyze a protocol without using the \texttt{cpsa4shapes} command, type:
\begin{verbatim}
$ cpsa4 -o prob.txt prob.scm
$ cpsa4graph -o prob.xhtml prob.txt
\end{verbatim}

The .xhtml results can be opened in a web browser.

\subsection{{\cpsa} Projects}\label{sec:cpsainit}

A {\cpsa} project is a directory containing some {\cpsa} input files
and items that orchestrate the use of the various {\cpsa} tools.  To
create a project, in a fresh directory, run \texttt{cpsa4init}.

\begin{verbatim}
$ mkdir project
$ cd project
$ cpsainit
Created Makefile
Created cpsa.mk
Created Make.hs
Created cpsa4prolog.pl
\end{verbatim}

The \texttt{cpsa4init} will not overwrite a file if it is already
present in the current directory.  The distribution provides two ways
to relieve users of the tedium of issuing individual commands.

\subsection{Using the {\cpsa} Makefile}

%% \begin{figure}
%%   \begin{quote}
%%     \begin{verbatim}
%% CPSAFLAGS = +RTS -M512m -RTS

%% SRCS := $(wildcard *.scm) $(wildcard *.lsp)

%% include cpsa.mk

%% all:    $(SRCS:%.scm=%_shapes.xhtml) $(SRCS:%.scm=%.xhtml) \
%%         $(SRCS:%.lsp=%_shapes.xhtml) $(SRCS:%.lsp=%.xhtml)

%% clean:
%%         -rm *.txt *.xhtml
%% \end{verbatim}
%%   \end{quote}
%%   \caption{\texttt{Makefile}}\label{fig:makefile}
%% \end{figure}

%% The easiest way to orchestrate {\cpsa} programs is to use GNU make.
%% The distribution comes with the file \texttt{cpsa4.mk} for inclusion
%% into your makefile.  Figure~\ref{fig:makefile} contains a sample
%% makefile. If you cut-and-paste, be sure to convert the leading spaces
%% in the last line into a tab character.  To analyze protocols, copy
%% these two files into a directory containing your protocol sources, and
%% type \texttt{make}.

The easiest way to orchestrate {\cpsa} programs is to use GNU make.  A
{\cpsa} project contains the files \texttt{Makefile} and
\texttt{cpsa4.mk}.  To analyze the problems in your project, type
\texttt{make}.  Edit the file \texttt{Makefile} to adjust program
flags to suit your needs.  For example, you many want to enable the
use of multiple processors during protocol analysis as described in
Section~\ref{sec:parallelism}.

The {\cpsa} program is Emacs friendly. If you run the above makefile
via \texttt{M-x compile}, the results will be displayed in a buffer in
Compilation Mode. The command \texttt{C-x `} will visit the locus of
the next error message or match (next-error) in your {\cpsa} input
file.

\subsection{Using the Haskell Makefile}

This approach is designed to be easy for Windows users, who do not want
to bother installing Cygwin or MSYS.

Locate \texttt{Make.hs} in your project directory. If using Windows,
double-click on the file and it will open up a new window with a
prompt. On a Mac or Linux machine, run \texttt{\$ ghci Make.hs}

From the \texttt{Make.hs} prompt, you can use the following commands:

\begin{itemize}
\item \texttt{build}: Run {\cpsa} on all protocols (.scm files) in the
  directory, and produce .xhtml output files displaying the results.
\item \texttt{clean}: Remove all {\cpsa} output files, to ensure that any
  changes to protocol files are reflected in the output.
\begin{itemize}
\item Because intermediate files are used for behind-the-scenes
  processing, it can be possible to have the results in the .xhtml
  output files not reflect the most up-to-date protocol file
  contents. If you make changes and don't see them reflected in the
  output, try running \texttt{clean}. Getting into the habit of
  running \texttt{clean} before \texttt{build} is a good idea.
\end{itemize}
\item \texttt{cpsa4} \texttt{"\textit{protocolname}"} Run {\cpsa} on just the
  protocol provided. Note that the file extension (.scm) should
  \textit{not} be included in the name; if your protocol is in
  \texttt{foo.scm}, you would run \texttt{cpsa4 "foo"}. Most useful if you
  have an exceptionally large number of protocols in a single
  directory.
\item \texttt{:q} Quit.
\end{itemize}

\subsection{Memory usage}

On large problems, {\cpsa} can consume large amounts of memory.  To
protect against memory exhaustion, run {\cpsa} with the command-line
options \verb|+RTS -M512m -RTS|.  The initial makefile includes these
options by default.

\subsection{Parallelism}\label{sec:parallelism}

CPSA is built so it can make use of multiple processors.  To make use
of more than one processor, start CPSA with a runtime flag that
specifies the number of processors to be used, such as
\verb|+RTS -N4 -RTS|.  The GHC documentation describes the \texttt{-N}
option in detail.
