\chapter{Constraining CPSA's search}
\label{ch:declarations}

If you have read Chapters~\ref{ch:basic} and~\ref{ch:algebra}, you
have seen the \texttt{uniq-orig} and \texttt{non-orig} features, which
are the first representatives of a large class of features called
\emph{declarations}.  These ``declare'' certain assumed properties about
an execution in order to constrain the tool's search.  In this chapter,
we will describe more precisely the notion of execution that motivates
the {\cpsa} analysis, and precisely define these and other declarations
available for use in the tool.

\section{Bundles: A Strand-Based Execution Model}
\label{sec:bundles}

\index{strand spaces} \index{event} The {\cpsa} tool is based on
strand space theory~\cite{DoghmiGuttmanThayer07}.  A strand is simply
a sequence of \emph{events}, which are transmissions or receptions of
messages.

A bundle is a set of strands, along with a satisfaction relation
$\rightarrow$ from transmissions to receptions, where for every
reception event $e_1$ of a message $m$, there is a unique transmission
event $e_0$ of $m$ such that $e_0 \rightarrow e_1$, and such that the
graph formed on the events of the strands with edges defined by
$\rightarrow$ and strand succession is acyclic.

Bundles express an explanatory framework in which a set of sequential
viewpoints of transmissions and receptions is self-contained.  Although
bundles do not totally order the events, they express the orderings that
are essential due to causality.

The most basic notion of a protocol is simply a set of roles which are
each themselves strands representing a template of behavior.  A bundle
is a bundle \emph{of a protocol} $P$ if every strand in the bundle is
either an instance of a role in $P$ or an instance of a penetrator
role.  There are two types of penetrator roles: derivation roles and
the ``create'' role.  The derivation roles are determined by the
message model; one matches each derivability rule in the algebra used.
The create role consists of a one-event strand in which certain basic
values may be transmitted.  This always includes all constants (such
as tag constants or the generator $g$ in the Diffie-Hellman algebra),
and also includes all values of basic sorts: the sorts other than
$\scap{mesg}$ in the basic algebra, and those sorts plus the
$\scap{rndx}$ sort in the Diffie-Hellman algebra.

\section{Secrecy assumptions}
\label{sec:secrecy_assumptions}
A priori, the analysis of (say) the initiator's point of view in
Needham-Schroeder is tasked with exploring all bundles of the
Needham-Schroeder protocol in which an initiator instance of
sufficient length is included.  As it turns out, there is a most
general shape, namely, the full-length initiator instance, alone.
However, the bundle this shape describes is not of great interest,
because the nonce chosen by the initiator appears in a ``create''
instance.

An assumption of secrecy is, in essence, a statement that the analyst
is interested only in certain bundles.  In the case of this nonce, for
instance, we assume the nonce is secret and are thus uninterested in
bundles in which that nonce appears in a ``create'' instance.

There are four declarations that may be used in {\cpsa} to represent
a secrecy assumption, but these assumptions are distinct from each other
and each have a particular semantic and syntactic meaning.

\index{carried subterm}
To describe these assumptions, we first need some terminology.  We
use ``carried'' to refer to subterms of a term that can potentially be
obtained from that term via decomposition. A pair carries each of its
elements, and an encryption carries its plaintext.  We extend this
notion transitively, so for instance $\enc{n1, n2}{K(a)}$ carries $n1$
since it carries the plaintext $(n1, n2)$ which in turn carries $n1$.

\index{origination} We say that a value \emph{originates} on a strand
when its first \emph{carried} occurrence is in a transmission.
\index{generation} We say that a value \emph{generates} on a strand
when its first occurrence (whether carried or not) in that strand is
in a transmission.

\begin{itemize}
\index{origination!unique}
\item A value \emph{uniquely originates} in a bundle if it originates
  on exactly one strand.  Note that if the unique origination point of
  a value is on a regular strand, then it cannot be produced in a
  ``create'' instance because that would constitute a second point of
  origination.

\index{origination!non-origination}
\item A value is \emph{non-originating} in a bundle if it does not
  originate on any strands.  A simple lemma can show that a value is
  non-originating only if it is not carried in any message in the
  bundle.

\index{origination!penetrator non-origination}
\item A value is \emph{penetrator non-originating} in a bundle of
  a protocol if it does not originate on any penetrator strands.
  Clearly, such a value cannot be produced in a ``create'' instance.
\end{itemize}

\ttindex{uniq-orig} \ttindex{non-orig} \ttindex{pen-non-orig} The
presence of a \texttt{uniq-orig}, \texttt{non-orig}, or
\texttt{pen-non-orig} declaration of a value is a statement by the
user that they are only interested in bundles in which the declared
value uniquely originates, is non-originating, or is penetrator
non-originating, respectively.

\ttindex{uniq-gen} The fourth declaration that imposes a secrecy
assumption is the \texttt{uniq-gen} declaration.  The presence of a
\texttt{uniq-gen} declaration of a value equates to a restriction to
bundles in which the declared value generates on only one strand.
Note that if a value that generates uniquely generates on a regular
strand, it cannot be produced in a ``create'' instance.

The \texttt{uniq-orig} (resp. \texttt{uniq-gen}) declaration can only
be used in {\cpsa} input for values that do originate (resp. generate)
on the role or in the skeleton in which the declaration appears.  This
forces the use of such declarations to create a secrecy assumption.

Because ``generation'' covers a wider set of occurrences than
``origination,'' a value that originates uniquely on one strand may
still generate on a different strand.  For instance, it may be used to
encrypt a message on that strand.  This means that one can model a key
(for instance) that is distributed from a point of unique origination,
and subsequently used to encrypt messages on other strands, without
modeling a particular mechanism for storing it in the recipient's
device.  This property of origination and generation can, however,
cause surprises if the key could surface unexpectedly in other
strands.  See Chapter~\ref{chap:channels:state} to understand how to
store values explicitly, when that is desired.

In fact, because ``generation'' covers a wider set of occurrences than
``origination,'' when a value $v$ originates on a strand, declaring
$v$ \texttt{uniq-gen} is actually stronger than declaring it
\texttt{uniq-orig}.  The \texttt{uniq-gen} declaration prohibits
unexplained key usage, as opposed to the \texttt{uniq-orig}
declaration that permits it.  

\begin{exercise}
Experiment with variants of your favorite simple protocol in which you
try replacing \texttt{uniq-orig} with \texttt{uniq-gen} and/or
\texttt{non-orig} with \texttt{pen-non-orig}.  What do you find?
\end{exercise}

\begin{exercise}
Suppose a protocol you wish to model involves key management, and
includes keys being encrypted and transmitted and received in
encrypted form.  What declaration should you use to model a long-term
(not recently chosen) key as secret, if you are concerned about the
actual possibility of that key leaking?
\end{exercise}

Note: these four declarations (\texttt{uniq-orig, non-orig,
  pen-non-orig,} and \texttt{uniq-gen}) may only be applied to basic
values.  One of the main purposes of such declarations is to restrict
the use of the ``create'' role in bundles, so such declarations on
values that cannot be produced in the ``create'' role is undesirable.

\section{Diffie-Hellman declarations}
A declaration arises as a part of the algorithm {\cpsa} uses for
Diffie-Hellman protocols.  Like other declarations, these may be
provided as part of a protocol or skeleton input.

\begin{itemize}

  \item \ttindex{absent} The \texttt{absent} declaration specifies
    that a particular \scap{rndx} variable does not occur in an % base or
    exponent expression.  This declaration is automatically added to
    specify $x$ is absent within $y$ whenever $y$ is an %  base or
    exponent expression that occurs in a strand in which $x$ is
    declared \texttt{uniq-gen} before the point where $x$ is
    generated.  This declaration is also added by the algorithm in
    rare circumstances.
%z
\end{itemize}

\iffalse

\section{Distinctness assumptions}
\label{sec:distinct_decls}

\index{inequality declarations}
\index{distinctness declarations}
One inconvenient consequence of representing protocol roles as strands
is that certain kinds of checks are not easy to represent.  The
structure of received terms implies that the participant executing the
protocol is capable of parsing and checking such formats.  However,
there is no way, with format, to represent a check that two values are
unequal.

The tool includes four special declarations that may be used to
describe distinctness of values.

\begin{itemize}

\ttindex{neq}
\item A pair of values may be declared unequal with the \texttt{neq}
  declaration.  For instance, \texttt{(neq (a b))} restricts the
  analysis to consider bundles in which $a \neq b$.  See
  \texttt{neq\_test.scm} in the examples directory for examples of use.

\ttindex{neqlist}
\item A larger list of values may be declared distinct with the
  \texttt{neqlist} declaration.  For instance, \texttt{(neqlist (a b c
    d))} adds a requirement that $a, b, c$, and $d$ all be distinct
  from one another.  This may be useful as shorthand, rather than the more
  cumbersome \texttt{(neq (a b) (a c) (a d) (b c) (b d) (c d))}.

\ttindex{subsort}
\item The $\scap{data}$ and $\scap{text}$ sorts are identical to one
  another, but they are distinct and thus allow a user to separate the
  purpose of certain parameters from others, and exclude analyses in
  which parameters are confused across types.  The \texttt{subsort}
  declaration may be used to sub-categorize values into disjoint
  classes, which may never be confused for each other.  For instance,
  an e-commerce protocol may involve prices, quantities, item numbers,
  et cetera.  Each subsort is named with a string.  For instance,
  \texttt{(subsort ("A" a1 a2 a3) ("B" b1 b2))} defines two classes of
  values, ``A'' and ``B'', and requires that no ``A'' value be equal
  to any other value with a subsort other than ``A''.  So in
  particular, all the $a$ variables are distinct from the $b$
  variables and vice versa, but the $a$ variables are not required to
  be distinct from each other.  See \texttt{subsort\_test.scm} in the
  examples directory for examples of use.

\ttindex{lt}
\item Finally, one particular reason two values may be unequal is if
  they are compared and put in an order.  For instance, in an auction,
  a bid may be rejected only if another bid is strictly higher.  Or, a
  value in a protocol model may be intended to represent time, and
  some of these may be strictly orderable with respect to each other.
  The \texttt{lt} declaration (short for ``less than'') asserts a
  strict comparability between two values in a partial order.  For
  instance, \texttt{(lt (a b))} declares that $b$ is less than $a$.
  The presence of \texttt{lt} declarations on pairs of values
  restricts the tool to bundles in which the implied orderings form no
  cycles.  See \texttt{lt\_test.scm} in the examples directory for
  examples of use.
\end{itemize}

For each of these declarations, {\cpsa} checks the conditions required
at every skeleton it considers.  Those skeletons for which any
required condition is violated are discarded.  You will never see a
skeleton with (for instance) \texttt{(neq (a a))} in it.

\section{Functional dependence assumptions}

Another disadvantage of the lightweight way {\cpsa} represents
protocols is that it does not understand the relationships between
values involved in messages.  For instance, an e-commerce protocol
may involve both identifiers for merchandise and prices.  An honest
merchant would assign a consistent price to any particular item,
but protocol roles and secrecy or distinctness assumptions cannot
describe this particular scenario.

\ttindex{fn-of} \index{functional dependence}
For this purpose, {\cpsa} allows the declaration of a functional
dependence using the \texttt{fn-of} declaration (short for ``function
of'').  Like the \texttt{subsort} declaration, a subtag is required
to identify the function, so that multiple independent functions could
be described if necessary.  An example:

\begin{center}
\verb|(fn-of ("f" (y x) (z w)))|
\end{center}

This declaration says that there is a function $f$ such that $y=f(x)$
and $z=f(w)$.  In the e-commerce example, if $f$ was the ``price of''
function, then $x$ and $w$ would be items and $y$ and $z$ would be
their respective prices.  The presence of a set of \texttt{fn-of}
declarations restricts the tool's search so that in this case, if
$x = w$ then $y = z$.

\index{Yahalom protocol}\index{examples!Yahalom}
The \texttt{fn-of} declaration is a powerful tool that can emulate
function symbols in the algebra.  See \texttt{yahalom.scm} in the
examples repository for an example protocol involving the \texttt{ltk}
function symbol.  We can totally emulate the use of this function symbol
via the \texttt{fn-of} declaration: see \texttt{fnof\_yahalom.scm} for how
this is done.

\index{init@\texttt{init}!use for introducing variables} It is important to note
that one limitation of the \texttt{fn-of} declaration is that it
cannot introduce variables not used in a skeleton or role.  The
emulation of the \texttt{ltk} symbol in \texttt{fnof\_yahalom.scm}, for
instance, requires three variables (the long-term key variable and the
two names) where there were formerly only two.  {\cpsa} will ignore
declarations involving variables not present anywhere, so in order to
make sure these declarations are not ignored, we need to use the
variables in the trace.  We use the \texttt{init} event (see
Chapter~\ref{chap:channels:state}) to introduce these variables, since we know
that without state events this will have no effect on the penetrator's
abilities.

\subsection{Equality constraints}

\index{equality} \ttindex{eq} The \texttt{fn-of} declaration allows a
user to express functional dependence relationships between values.
In simpler inputs, it may be desirable to simply declare that two
values are equal; this is particularly useful for Diffie-Hellman
protocols, for instance, to declare that two separate parties agree on
a session key.  The \texttt{eq} declaration simply declares that two
values must be equal.  An example:

\begin{center}
  \verb|(eq ((exp (gen) (mul x y)) (exp (gen) (mul z w))))|
\end{center}

This declaration guarantees that $g^{xy} = g^{zw}$.

\section{Role declarations and conditional role declarations}
\label{sec:role_decls2}

\index{role declarations} When a declaration is present in a
\texttt{defskeleton} in an input file, the constraint is applied to
all analyses that descends from that skeleton.  In
Section~\ref{sec:blanchet} we discussed the notion of a role
declaration and how such a declaration differs in meaning from a
skeleton declaration.

Here we discuss more precisely when a role declaration affects a
skeleton.  Skeletons are made up of \emph{instances}, which follow
the structure of a role, but instances do not have to be full-length.
The declarations inherited in an instance of a role from the role
may vary based on the length of the instance.

Declarations are, most importantly, declarations on a term or list of
terms.  A declaration is not inherited in an instance of height $h$
(that is, an instance of the first $h$ events in the role) unless all the
variables present in terms in the declaration are present in the first $h$
events in the role's trace.

The tool also regards declarations as about a node or list of nodes,
although in most cases that list is empty.  There are two exceptions.
\ttindex{uniq-gen} \ttindex{uniq-orig} The \texttt{uniq-gen} and
\texttt{uniq-orig} declarations refer to the node at which the
declared value generates or originates, but in an automatic and
hidden way.

Role declarations refer to a node which is a particular event in the
role's trace.  In addition to the rule about variables in a
declaration's terms, a declaration is only inherited if all its nodes
are present in the instance.

\index{role declarations!conditional} All the declarations natively
included in the tool may include a node when used in a role.  This
allows the user to describe declarations that are conditional on
sufficient progress being made through the role's trace.  This may
be appropriate when, for instance, a participant receives a certificate
and comes to trust a party it has already started communicating with.

A declaration is made conditional by replacing a term (or list of
terms) with a list whose first element is the term (or list of terms)
and whose second element is the node.  Most {\cpsa} declarations allow
many declarations to be made in one S-expression; for instance
\texttt{(non-orig k1 k2 k3)} is actually three distinct declarations
rather than one.  To make the $k1$ declaration conditional on reaching
node 3, we would write \texttt{(non-orig (k1 2) k2 k3)}.  See
Table~\ref{tab:decl_syntax} for full detail on the syntax for
conditional declarations.

\section{Diffie-Hellman declarations}
There are two declarations that arise as a part of the algorithm
{\cpsa} uses for Diffie-Hellman protocols.  Like other declarations,
these may also be provided as part of a protocol or skeleton input.

\begin{itemize}

  \item \ttindex{absent} The \texttt{absent} declaration specifies
  that a particular \scap{rndx} variable does not occur in an % base or
  exponent expression.  This declaration is automatically added to
  specify $x$ is absent within $y$ whenever $y$ is an % base or
  exponent expression that occurs in a strand in which $x$ is declared
  \texttt{uniq-gen} before the point where $x$ is generated.  This
  declaration is also added by the algorithm in rare circumstances.

  \item \ttindex{precur} The \texttt{precur} declaration specifies
    that a particular node is present in the skeleton to be the
    ``precursor'' for some other value.  This is used to prevent
    {\cpsa} from failing to terminate in certain cases.
\end{itemize}

\section{Other declarations}
\label{sec:other_decls}

The tool makes use of declaration-like syntax for several other
purposes.

\paragraph{Precedes.}
\ttindex{precedes} In {\cpsa} text output, you may notice an
S-expression that starts with \texttt{precedes}.  This is how {\cpsa}
records the orderings present in a skeleton, before it is graphed.
You may also use precedes declarations within a \texttt{defskeleton},
to set up orderings between strands.

\ttindex{leadsto}
\paragraph{Leadsto.}  Similarly, the \texttt{leadsto} field of
a \texttt{defskeleton} specifies state causality orderings.  See
Chapter~\ref{chap:channels:state} for more detail on what this means.

\paragraph{Priority.}
\ttindex{priority} The tool contains an ability to declare a priority
for certain receptions that differs from the default.  Priority takes
precedence over all other search orderings.  See
Section~\ref{sec:decl_syntax} for the format requirements for
declaring priorities, and the \texttt{priority\_test.scm} example in the
examples directory.

Note that the default priority is 5, and priority 0 indicates that the
tool should never bother solving tests at those nodes.  This may be of
use, for instance, if solving one particular node leads to infinite
analysis, but other nodes would result in a quick determination that a
skeleton is dead.

\paragraph{User-defined declarations.}
The final type of declaration supported by {\cpsa} is totally general,
but does not affect the search.  Instead, this type of declaration is
merely kept around as a note about the skeleton in question, but a
note that is aware of its use of algebra variables and nodes and
evolves as the skeleton evolves.

\index{user-defined declarations} \ttindex{decl} This feature is meant
for advanced users only, who may wish to write their own custom
post-processing tools.  For this reason we do not give an example
involving user-defined declarations; for the proper syntax, consult
Table~\ref{tab:decl_syntax}.  The keyword to identify a user-defined
declaration is \texttt{decl}.

\fi

\section{A Caution about typos}
Declarations are typically found syntactically in ``association
lists'' which include declarations and may also include other fields,
whether in protocol declarations, role declarations within them, or in
skeleton query declarations.  For instance, the {\cpsa} tool creates
associations in its \texttt{defskeleton} outputs specifying things
like the algorithmic method used to produce this skeleton, or the
label of the parent, etc.  {\cpsa} handles comments in the same way.
Any association list entry \verb|(header| \dots
\emph{stuff}\dots\verb|)| in which the header is unrecognized is
treated as a comment, and preserved in the output.  This is convenient
not just for documenting your specifications, but also for defining
material to be used not by {\cpsa} itself but by downstream tools.

Be careful when using declarations that you do not mis-spell the
keywords!  If you do, they will typically be treated just like a
comment and ignored, so they will not have the effect you intended.

%%% Local Variables:
%%% mode: latex
%%% TeX-master: "cpsa4manual"
%%% End:
