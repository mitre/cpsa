\chapter{CPSA Tools}
\label{ch:tools}

This chapter describes programs that complement the {\cpsa} program.
All the tools accept a \texttt{--help} option that describes the
tool's usage and a \texttt{--version} option that prints the version
of the tool.

\begin{description}
\item[\texttt{cpsa4init}:] Populates a directory with files used to
  create a {\cpsa} project.  See Section~\ref{sec:cpsainit}.

\item[\texttt{cpsa4shape}:] Cuts down {\cpsa}'s output to only show
  final results---the shapes.

\item[\texttt{cpsa4graph}:] Constructs an \textsc{xhtml} document that
  visualizes the output of a {\cpsa} run or the output of
  \texttt{cpsa4shape}.

\item[\texttt{cpsa4goalsat}:]  Prints the labels of skeletons that do
  not satisfy a \texttt{defgoal}.  See
  Section~\ref{sec:goalsoverview}.  For each point of view skeleton
  specified by a goal, the program prints a list.  The first element
  of the list is the label of the point of view skeleton.  The
  remaining elements of the list are labels of the skeletons that do
  not satisfy the goal.

\item[\texttt{cpsa4prot}:] Translates protocols expressed in Alice and
  Bob notation into {\cpsa}'s \texttt{defprotocol} syntax.  It
  translates \texttt{defprot} forms, and passes all else through
  unchanged.  The syntax of \texttt{defprot} forms is presented in
  Table~\ref{tab:defprot}.

  The basic layout of a \texttt{defprot} is a chronological
  description of things that happen in the intended execution of the
  protocol.  The s-expression \texttt{(msg A B M)} is used to indicate
  that role \texttt{A} sends \texttt{M}, while role \texttt{B}
  receives it.  Other features of \texttt{cpsa4prot} are given in Table~\ref{tab:defprot_commands}.

  See Figure~\ref{fig:defprot} for an
  example.

\item[\texttt{cpsa4sas}:]  Generates Shape Analysis Sentences from the
  output of {\cpsa}.  See Section~\ref{sec:sas}.

\item[\texttt{cpsa4pp}:] Pretty prints {\cpsa} input.  It can also be
  used to translate S-expression syntax into \textsc{json} for use
  with tools written in other languages such as Python.

\item[\texttt{cpsa42latex}:] Translates {\cpsa} macros specifying
  message formats into equivalent {\LaTeX} macros. Algebra primitives
  such as \texttt{enc} and \texttt{pubk} are mapped to unspecified
  {\LaTeX} macros with the same name and arity (sometimes with an
  optional tag), allowing the user to easily customize how they would
  like these primitives to be typeset.

\item[\texttt{cpsa4json}:]  Translates {\cpsa} input in \textsc{json}
  syntax into S-expression syntax.

\item[\texttt{cpsa4diff}:]  Compares the skeletons in two files.  If a
  difference is detected, it prints the first difference and exits.

\end{description}

\begin{table}
\begin{center}\scshape
\begin{tabular}{rcl}
prot&$\leftarrow$&(\sym{defprot} id alg vars form$\ast$ rule$\ast$ prot-alist)
\\ vars&$\leftarrow$&(\sym{vars} vdecl$\ast$)
\\ form&$\leftarrow$&(\sym{msg} role role chmsg)
\\ &$\mid$&(\sym{msg} role role chmsg chmsg)
\\ &$\mid$&(\sym{from} role chmsg)
\\ &$\mid$&(\sym{to} role chmsg)
\\ &$\mid$&(\sym{assume} role role-alist)
\\ &$\mid$&(\sym{clone} role role)
\\ chmmsg&$\leftarrow$&(\sym{chmsg} id term) $\mid$ (\sym{chmsg} term)
$\mid$ term
\end{tabular}
\end{center}
\caption{\texttt{defprot} Grammar}\label{tab:defprot}
\end{table}

\begin{table}
  \begin{center}
    \begin{tabular}{c|c}
      Form & Meaning \\
      \hline
      (\texttt{msg A B M}) & \texttt{A} sends \texttt{M} to \texttt{B} \\
      (\texttt{msg A B M N}) & \texttt{A} sends \texttt{M} and \texttt{B} receives it as \texttt{N}\\
      (\texttt{from A M}) & \texttt{A} sends \texttt{M}\\
      (\texttt{to A M}) & \texttt{A} receives \texttt{M}\\
      (\texttt{assume A M}) & Add the assumption \texttt{M} to the role \texttt{A}\\
      (\texttt{clone A B}) & Create a role \texttt{B} which includes all behavior described for \texttt{A} to this point.
    \end{tabular}
  \end{center}
  \caption{\texttt{defprot} commands}\label{tab:defprot_commands}
  \end{table}

\begin{figure}
  \begin{quote}
    \begin{verbatim}
$ cat blanchet.prot
(defprot blanchet basic
  (vars (a b akey) (s skey) (d data))
  (msg init resp (enc (enc s (invk a)) b))
  (msg resp init (enc d s))
  (assume init (uniq-orig s))
  (assume resp (uniq-orig d))
  (comment "Blanchet's protocol"))
$ cpsa4prot blanchet.prot
(defprotocol blanchet basic
  (defrole init
    (vars (a b akey) (s skey) (d data))
    (trace (send (enc (enc s (invk a)) b)) (recv (enc d s)))
    (uniq-orig s))
  (defrole resp
    (vars (a b akey) (s skey) (d data))
    (trace (recv (enc (enc s (invk a)) b)) (send (enc d s)))
    (uniq-orig d))
  (comment "Blanchet's protocol"))
$
    \end{verbatim}
  \end{quote}
\caption{\texttt{defprot} Example}\label{fig:defprot}
\end{figure}
