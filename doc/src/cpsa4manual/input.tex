\chapter{CPSA Program}
\label{ch:input}

\section{CPSA pre-processing}

The {\cpsa} tool performs a pre-processing step before it interprets
its input.  There are two important features that take place during
pre-processing: macros and file inclusion.

\paragraph{File inclusion.}
\ttindex{include} {\cpsa} input files that become large enough become
unweildy, and the user may wish to break them down into logical
components and include one file in another.  For instance, one might
wish to separate protocol definitions out so they can be reused, or
a user might wish to put together a file of macros they find useful.
To include one file in another, add \texttt{(include "filename")}
as a top-level S-expression where you wish that file to be included.
Inclusion recognizes only relative paths.

\paragraph{Macros.}
\ttindex{defmacro} \index{macros} The {\cpsa} pre-processor interprets
macros.

\index{examples!Envelope protocol!macros in}
To define a macro in a {\cpsa} input file, use the \texttt{defmacro}
keyword in an S-expression.  The \texttt{envelope.scm} file in the
examples directory uses the following macro:

\begin{verbatim}
;; This is the refusal token
(defmacro (refuse n pcr v k aik)
  (enc "quote" (extend "refuse" (extend n pcr)) (enc v k) aik))
\end{verbatim}

The first input to the defmacro defines the format that will trigger
the macro.  In this case, the first macros are defined for an
S-expression with keyword \texttt{refuse} and five additional inputs.
The second input to the defmacro describes what the macro should be
replaced with.  Symbols that exactly match the subsequent symbols in
the first input are interpreted as standing for the inputs when the
macro is used.  So for instance \texttt{(refuse a b c d e)} would be
replaced by

\begin{verbatim}
(enc "quote" (extend "refuse" (extend a b)) (env c d) e)
\end{verbatim}

\noindent
wherever it appears in what follows.  \index{macros!nesting} Macros
can call on other macros, but there is a depth limit to the amount of
recursion that this can entail.  In the example, \texttt{extend} is
actually another macro.

\ttindexalt{\^}{(macro splicing)}
\index{macros!splicing} Normally a \texttt{defmacro} will
replace a symbol with a single S-expression, but the \texttt{\^}
(splice) keyword can be used to indicate that a macro should be
replaced with more than one S-expression.  This may be of use, for
instance, to describe a portion of a role's trace, when defining
multiple roles with some behavior in common.  For instance:

\begin{verbatim}
(defmacro (handshake n a b)
  (^ (send (enc "hello" a b n (pubk b)))
     (recv (enc "hello-received" a n (pubk a))))
\end{verbatim}

Note that the pre-processor actually handles the splice keyword as a
separate pre-processing step after macro expansion.  For this reason,
use of \texttt{\^} outside of macros can produce unanticipated behavior.

\section{CPSA input syntax}
\label{sec:input}

The complete syntax for the analyzer using the Basic Crypto Algebra is
shown in Table~\ref{tab:syntax}.  The start grammar symbol is
\textsc{file}, and the terminal grammar symbols are: \textsc{(, ),
  symbol, string, integer,} and the constants set in typewriter font.

The \textsc{alist}, \textsc{prot-alist}, \textsc{role-alist},
and \textsc{skel-alist} productions are Lisp style association lists,
that is, lists of key-value pairs, where every key is a symbol.
Key-value pairs with unrecognized keys are ignored, and are available
for use by other tools.  On output, unrecognized key-value pairs are
preserved when printing protocols, but elided when printing skeletons.

\begin{table}
\begin{center}\scshape
\begin{tabular}{rcl}
file&$\leftarrow$&herald?~form+
\\herald&$\leftarrow$&
(\sym{herald}~$[\mbox{symbol}\mid\mbox{string}]$~alist)
\\form&$\leftarrow$&(\sym{comment}~$\ldots)\mid\mbox{protocol}\mid\mbox{skeleton}\mid\mbox{goal}$
\\ protocol&$\leftarrow$&
(\sym{defprotocol} id alg role+ rule$\ast$ prot-alist)
\\ id&$\leftarrow$&symbol
\\ alg&$\leftarrow$&\sym{basic}~$\mid$~\sym{diffie-hellman}
\\ role&$\leftarrow$&
(\sym{defrole} id vars trace role-alist)
\\ vars&$\leftarrow$&
(\sym{vars} vdecl$\ast$)
\\ vdecl&$\leftarrow$&(id+ sort)
\\ trace&$\leftarrow$&(\sym{trace} \{event $\mid$ annot\}+)
\\ event&$\leftarrow$&({dir id? term})
\\ dir&$\leftarrow$&$\sym{send}\mid\sym{recv}\mid\sym{load}\mid\sym{stor}$
\\ annot&$\leftarrow$&(\sym{cheq} id term) $\mid$
                       (\sym{rely} conclusion) $\mid$ (\sym{guarantee} conclusion)
\\ rule&$\leftarrow$&(\sym{defrule} id sentence alist)
\\ role-alist&$\leftarrow$&role-decl\mbox{ role-alist }$\mid$~\mbox{alist role-alist}
\\ alist&$\leftarrow$&(symbol~$\ldots$)?\mbox{ alist?}
\\ skeleton&$\leftarrow$&
(\sym{defskeleton} id vars strand+ skel-alist)
\\ strand&$\leftarrow$& (\sym{defstrand} id integer maplet$\ast$)
\\ &$\mid$&(\sym{defstrandmax} id integer? maplet$\ast$)
\\ &$\mid$&(\sym{deflistener} term)
\\ maplet&$\leftarrow$&
(term term)
\\ skel-alist&$\leftarrow$&\mbox{skel-decl skel-alist }$\mid$~\mbox{alist skel-alist}
\\ &$\mid$&$(\sym{precedes}\mbox{ node-pair}\ast)\mbox{ skel-alist}$
\\ &$\mid$&$(\sym{leadsto}\mbox{ node-pair}\ast)\mbox{ skel-alist}$
\\ &$\mid$&(\sym{goal} sentence+)$\mid$~$(\sym{facts} (\mbox{id term*}))$
\\ node-pair&$\leftarrow$&(node node)
\\ node&$\leftarrow$&(integer integer)
\\ goal&$\leftarrow$&(\sym{defgoal} id sentence+ alist)
\end{tabular}
\end{center}
\caption[{\cpsa} Input Syntax]{{\cpsa} Input Syntax.  See
  Tables~\ref{tab:basic_term} and~\ref{tab:dh_term} for algebra syntax
  (for the $\scap{term}$ and $\scap{sort}$ symbols),
  Table~\ref{tab:decl_syntax} for declaration syntax (for the
  $\scap{role-decl}$ and $\scap{skel-decl}$ symbols), and
  Table~\ref{tab:goal_syntax} for goal syntax (for the
  $\scap{sentence}$ and $\scap{conclusion}$ symbols).}
\label{tab:syntax}
\end{table}

The contents of a file can be interpreted as a sequence of
S-expressions.  The S-expressions used are restricted so that most
dialects of Lisp can read them, and characters within symbols and
strings never need quoting.  Every list is proper.  An S-expression
atom is either a \textsc{symbol}, an \textsc{integer}, or a
\textsc{string}.  The characters that make up a symbol are the
letters, the digits, and the special characters in
``\verb|-*/<=>!?:$%_&~^+|''.  A symbol may not begin with a digit or a
sign followed by a digit.  The characters that make up a string are
the printing characters omitting double quote and backslash, except
when double quote and backslash are escaped using the backslash
character.  Double quotes delimit a string.  A comment\index{comments}
begins with a semicolon, or is an S-expression list at top-level that
starts with the \texttt{comment} symbol.

\section{Algebra reference}\label{sec:algebra_ref}

\subsection{Basic crypto algebra}

\begin{table}
\centering
%
\begin{tabular}{|lll|}\multicolumn{3}{c}{Sorts}\\ \hline
  Sorts:
  & \multicolumn{2}{l|}{\scap{name}, \scap{text},
    \scap{data}, \scap{tag}, \scap{skey}, \scap{akey}  $<$
    \scap{mesg}} \\
  &  \multicolumn{2}{l|}{\scap{chan}, \scap{locn} $<$ \scap{mesg}} \\
  \hline \multicolumn{3}{c}{Operations}\\ \hline
  $\enc{\cdot}{(\cdot)}\phantom{\colon}$ &
  $\scap{mesg}\times\scap{mesg}\rightarrow\scap{mesg} $&
  Encryption\\ $(\cdot,\cdot)\phantom{\colon}$ &
  $\scap{mesg}\times\scap{mesg}\rightarrow\scap{mesg}$
  &Pairing\\
  $\#(\cdot)\phantom{\colon}$ & $\scap{mesg} \rightarrow \scap{mesg}$ & Hashing \\
  $K_{(\cdot)}\phantom{\colon}$ &
  $\scap{name}\rightarrow\scap{akey}$ &Public key of
  name\\ $K^{s}_{(\cdot)}\phantom{\colon}$ &
  $\scap{name}\rightarrow\scap{akey}$ & $s$-Public key of
  name\\ $(\cdot)^{-1}\phantom{\colon}$ &
  $\scap{akey}\rightarrow\scap{akey}$ &Inverse of
  key\\ $\cn{ltk}(\cdot,\cdot)\phantom{\colon}$ &
  $\scap{name}\times\scap{name}\rightarrow\scap{skey}$ & Long-term key\\
  \hline
%
\multicolumn{3}{c}{Constants}\\ \hline
$Tags$ $\phantom{\colon}$ & $\scap{tag}$ & Tag constants \\ \hline
\multicolumn{3}{c}{Equations}\\ \hline
%
$(a^{-1})^{-1} = a$ & $a \colon \scap{akey}$ &  \\ \hline
\multicolumn{3}{c}{Derivations}\\ \hline
$m_0, m_1 \vDash \enc{m_0}{m_1}$ & $m_0, m_1 \colon \scap{mesg}$ & Encryption \\
$m_0, m_1 \vDash (m_0, m_1)$ & $m_0, m_1 \colon \scap{mesg}$ & Pairing \\
$(m_0, m_1) \vDash \{m_0, m_1\}$ & $m_0, m_1 \colon \scap{mesg}$ & Destructuring \\
$\enc{m}{k}, \mathrm{inv}(k) \vDash m$ & $m, k \colon \scap{mesg}$ & Decryption \\
$m \vDash \#(m)$ & $m \colon \scap{mesg}$ & Hashing\\
\hline
%
\end{tabular}
\caption{The Basic Cryptoalgebra}
\label{tab:basic_algebra_signature}
\end{table}

The basic crypto algebra is an order-sorted algebra with signature
described in Table~\ref{tab:basic_algebra_signature}.  The algebra is
the free order-sorted algebra generated by the function symbols,
sorts, and constants given, modulo the one equation.

Additionally, {\cpsa} reasons about derivability of values from other
values, and the basic derivation rules are given in the table, where
$\mathrm{inv}$ is defined as follows (with $\bot$ indicating
``undefined''):
\begin{equation}
%
\mathrm{inv}(a) = \left\{
%
\begin{array}{ll}
%
a^{-1} & \mbox{if $a : \scap{akey}$}\\
%
a & \mbox{if $a$ is not a variable of sort $\scap{mesg}$} \\
%
\bot & \mbox{otherwise}.
%
\end{array}
%
\right.  %}
\end{equation}

Because $\mathrm{inv}$ is undefined on variables of sort
$\scap{mesg}$, {\cpsa} cannot handle protocols or skeletons in which a
value is encrypted with such a variable.  This is a consequence of the
choice we made in the design of {\cpsa} to use only one encryption
function symbol, despite there being two forms of encryption
(symmetric and asymmetric).  When encrypting with a variable of sort
$\scap{mesg}$, the type of encryption is ambiguous.

Table~\ref{tab:basic_term} describes the grammar for basic crypto
algbera messages in the input syntax.

\begin{table}
\centering{\scshape
\begin{tabular}{rcl}
\\ alg&$\leftarrow$&$\sym{basic}$
\\
  sort&$\leftarrow$&$\sym{text}\mid\sym{data}\mid\sym{name}\mid\sym{tag}\mid\sym{skey}\mid\sym{akey}$\\
  &$\mid$&$\sym{chan}\mid\sym{locn}\mid\sym{mesg}\mid$ id
\\ id&$\leftarrow$&symbol
\\ term&$\leftarrow$&
$\mbox{id}\mid(\sym{pubk}\mbox{ id})
\mid(\sym{privk}\mbox{ id})
\mid(\sym{invk}\mbox{ id})$
\\ &$\mid$&$(\sym{pubk}\mbox{ id}\mbox{ string})
\mid(\sym{privk}\mbox{ id}\mbox{ string})$
\\ &$\mid$&$(\sym{ltk}\mbox{ id id})\mid\mbox{string}\mid(\sym{cat}\mbox{ term+})$
\\ &$\mid$&$(\sym{enc}\mbox{ term+ term})\mid(\sym{hash}\mbox{ term+})$
\\ &$\mid$&$(\mbox{id term+})$
\end{tabular}}
\caption{{\cpsa} Basic Algebra Syntax; {sorts}, operators \textsc{id}
  from \textsc{Lang} field}\label{tab:basic_term}
\end{table}

Each of these function symbols has a specific interpretation in the
algebra signature; for instance, \texttt{enc} refers to the
$\enc{\cdot}{\cdot}$ function symbol, and \texttt{pubk} refers to the
$K_{(\cdot)}$ function symbol when it has one input, but to
$K^s_{(\cdot)}$ when it has two inputs, where the second is $s$.
\texttt{privk} refers to the composition of either $K_{(\cdot)}$ or
$K^s_{(\cdot)}$ with $(\cdot)^{-1}$.

\subsection{The Diffie-Hellman crypto algebra}
\label{sec:input:dh}

The Diffie-Hellman crypto algebra is an order-sorted algebra with signature
described in Table~\ref{tab:dh_algebra_signature}.  The algebra is
the free order-sorted algebra generated by the function symbols,
sorts, and constants given, modulo the equations.

\begin{table}
\centering
%
\begin{tabular}{|lll|}
\multicolumn{3}{c}{Sorts}\\ \hline
Sorts: & \multicolumn{2}{l|}{\scap{name}, \scap{text}, \scap{data},
         \scap{tag}, \scap{skey}, \scap{akey}%, \scap{base}
         $<$ \scap{mesg}} \\
       & \multicolumn{2}{l|}{\scap{rndx} $<$ \scap{expt} $<$ \scap{mesg}} \\
  &  \multicolumn{2}{l|}{\scap{chan}, \scap{locn} $<$ \scap{mesg}}
\\ \hline \multicolumn{3}{c}{Operations}\\ \hline
$\enc{\cdot}{(\cdot)}\phantom{\colon}$ &
$\scap{mesg}\times\scap{mesg}\rightarrow\scap{mesg} $& Encryption\\
$(\cdot,\cdot)\phantom{\colon}$ &
$\scap{mesg}\times\scap{mesg}\rightarrow\scap{mesg}$ &Pairing\\
$\#(\cdot)\phantom{\colon}$ & $\scap{mesg} \rightarrow \scap{mesg}$ & Hashing \\
$K_{(\cdot)}\phantom{\colon}$ &
$\scap{name}\rightarrow\scap{akey}$ &Public key of name\\
$K^{s}_{(\cdot)}\phantom{\colon}$ &
$\scap{name}\rightarrow\scap{akey}$ & $s$-Public key of name\\
$(\cdot)^{-1}\phantom{\colon}$ & $\scap{akey}\rightarrow\scap{akey}$ &Inverse of key\\
$\cn{ltk}(\cdot,\cdot)\phantom{\colon}$ &
$\scap{name}\times\scap{name}\rightarrow\scap{skey}$ & Long-term key \\
$\cn{bltk}(\cdot,\cdot)\phantom{\colon}$ &
$\scap{name}\times\scap{name}\rightarrow\scap{skey}$ & Bi-directional LTK \\
$(\cdot)^{(\cdot)}\phantom{\colon}$ & $\scap{base} \times \scap{expt} \rightarrow \scap{base}$ &
Exponentiation \\
$(\cdot \cdot)\phantom{\colon}$ & $\scap{expt} \times \scap{expt} \rightarrow \scap{expt}$ &
Multiplication \\
$i(\cdot) \phantom{\colon}$ & $\scap{expt} \rightarrow \scap{expt}$ & Mult. Inverse\\
\hline
%
\multicolumn{3}{c}{Constants}\\ \hline
$Tags$ $\phantom{\colon}$ & $\scap{mesg}$ & Tag constants \\
$g \phantom{\colon}$ & $\scap{base}$ & Generator \\
$1 \phantom{\colon}$ & $\scap{expt}$ & Mult. Identity \\ \hline

\multicolumn{3}{c}{Equations}\\ \hline
%
$(a^{-1})^{-1} = a$ & & $a \colon \scap{akey}$\\
$\cn{bltk}(a,b) = \cn{bltk}(b,a)$ & & $a, b \colon \scap{name}$ \\
$xy = yx$ & $x(yz) = (xy)z$ & $x, y \colon \scap{expt}$\\
$x1 = x$ & $x(i(x)) = 1$ & $x \colon \scap{expt}$ \\
$h^1 = h$ & $(h^x)^y = h^{(xy)}$ & $h \colon \scap{base}, x, y \colon \scap{expt}$ \\
\hline
\multicolumn{3}{c}{Derivations}\\ \hline
$m_0, m_1 \vDash \enc{m_0}{m_1}$ & $m_0, m_1 \colon \scap{mesg}$ & Encryption \\
$m_0, m_1 \vDash (m_0, m_1)$ & $m_0, m_1 \colon \scap{mesg}$ & Pairing \\
$(m_0, m_1) \vDash \{m_0, m_1\}$ & $m_0, m_1 \colon \scap{mesg}$ & Destructuring \\
$\enc{m}{k}, \mathrm{inv}(k) \vDash m$ & $m, k \colon \scap{mesg}$ & Decryption \\
$m \vDash \#(m)$ & $m \colon \scap{mesg}$ & Hashing\\
$x, y \vDash xy$ & $x, y \colon \scap{expt}$ & Multiplication\\
$x \vDash i(x)$ & $x \colon \scap{expt}$ & Inversion\\
$h, x \vDash h^x$ & $h \colon \scap{base}, x \colon \scap{expt}$ & Exponentiation\\
\hline
%
\end{tabular}
%\caption[]{The Diffie-Hellman Cryptoalgebra, continued.}
%\label{tab:dh_algebra_signature2}
\caption{The Diffie-Hellman Cryptoalgebra}
\label{tab:dh_algebra_signature}
\end{table}

\ttindex{expt} In Section~\ref{sec:dh} we discussed modeling
Diffie-Hellman in {\cpsa}, but we said almost nothing about the
$\scap{expt}$ sort that features prominently in the algebra signature.
Our message model for Diffie-Hellman has two sorts of exponents.  The
$\scap{rndx}$ sort is for exponents that are to be thought of as
atomic values.  Randomly chosen exponents should always be of this
sort, which is why \texttt{rndx} appears exclusively in our examples,
while \texttt{expt} does not.  The $\scap{expt}$ sort is for compound
expressions involving exponents, or values \texttt{(exp (gen) alpha)}
that are received with unconstrained exponent.  There may also be
variables of this sort, but they should not first occur in
transmissions in protocol roles.
%   ,
%   and even using them in a \texttt{defskeleton} is unusual.
%   However, they do appear in the output with some frequency, due to one
%   of two kinds of bits of reasoning: first, a base variable (say, $h$)
%   may be rewritten as an explicit power of $g$.  When this happens,
%   since $h$ could stand for any power of $g$ including a power with a
%   complex variable, if we rewrite $h$ as $g^z$, $z$ will need to be of
%   the generic $\scap{expt}$ sort so that those possibilities are
%   covered.  Second, {\cpsa} will sometimes assume the presence of a
%   value in some instance related to another one, and a generic
%   $\scap{expt}$ variable is used to capture the somewhat arbitrary
%   relationship between the two.

Table~\ref{tab:dh_term} describes the grammar for Diffie-Hellman
crypto algbera messages in the input syntax.  Symbols in the grammar
appearing in the grammar for the basic crypto algebra retain their
same interpretation.  Additionally, the \texttt{gen} and \texttt{one}
symbols, used as 0-ary function symbols, specify the $g$ and $1$
constants, respectively.

\begin{table}
\centering{\scshape
\begin{tabular}{rcl}
\\ alg&$\leftarrow$&$\sym{diffie-hellman}$
\\
  sort&$\leftarrow$&$\sym{text}\mid\sym{data}\mid\sym{name}\mid\sym{tag}\mid\sym{skey}\mid\sym{akey}$
  \\ &$\mid$ & $\sym{rndx}~\mid\sym{expt}\mid%\sym{base}\mid
               \sym{chan}\mid\sym{locn}\mid\sym{mesg}\mid$ id
\\ id&$\leftarrow$&symbol
\\ term&$\leftarrow$&
$\mbox{id}\mid(\sym{pubk}\mbox{ id})
\mid(\sym{privk}\mbox{ id})
\mid(\sym{invk}\mbox{ id})$
\\ &$\mid$&$(\sym{pubk}\mbox{ id}\mbox{ string})
\mid(\sym{privk}\mbox{ id}\mbox{ string})$
\\ &$\mid$&$(\sym{ltk}\mbox{ id id})\mid\mbox{string}\mid(\sym{cat}\mbox{ term+})$
\\ &$\mid$&$(\sym{enc}\mbox{ term+ term})\mid(\sym{hash}\mbox{ term+})$
\\ &$\mid$&$(\sym{exp}\mbox{ term term })\mid(\sym{gen})\mid(\sym{mul}\mbox{ term term})$
\\ &$\mid$&$(\sym{rec}\mbox{ term})\mid (\mbox{id term+})$
\end{tabular}}
\caption{{\cpsa} Diffie-Hellman Algebra Syntax}\label{tab:dh_term}
\end{table}
\section{Declaration syntax}
\label{sec:decl_syntax}

Table~\ref{tab:decl_syntax} gives a grammar for the syntax of native
and user-defined declarations.  For the purposes of this table, there
are two top-level symbols: \textsc{role-decl} and \textsc{skel-decl}.

\begin{table}
\begin{center}\scshape
  \begin{tabular}{rcl}
\\ role-decl&$\leftarrow$&(one-term-decl ht-term+)
%   \\ &$\mid$&(two-term-decl ht-termpair+)
%   \\ &$\mid$&(\sym{neqlist} ht-termlist+)
%   \\ &$\mid$&(\sym{subsort} ht-subsort+)
%   \\ &$\mid$&(\sym{fn-of} ht-function+)
%   \\ &$\mid$&(\sym{decl} symbol ht-general+)
%\\ &$\mid$&(\sym{critical-sections} (integer integer)+)
\\ &$\mid$&(\sym{priority} (height integer)+)
\\ &$\mid$&(\sym{facts} (id term*)*)~$\mid$~(\sym{assume} conclusion*)
\\ one-term-decl&$\leftarrow$&\sym{non-orig}~$\mid$~\sym{pen-non-orig}~$\mid$~\sym{uniq-orig}~$\mid$~\sym{uniq-gen}
\\ &$\mid$&\sym{conf}~$\mid$~\sym{auth}~$\mid$~\sym{gen-st}
%\\ two-term-decl&$\leftarrow$&\sym{absent}  % \sym{neq}~$\mid$~\sym{lt}
\\ height&$\leftarrow$&integer
\\ ht-term&$\leftarrow$&term~$\mid$~(term height)
%   \\ ht-termpair&$\leftarrow$&(term term)~$\mid$~(term term height)
%   \\ ht-termlist&$\leftarrow$&(term+)~$\mid$~(term+ height)
%   \\ ht-subsort&$\leftarrow$&(aux ht-term+)
%   \\ ht-function&$\leftarrow$&(aux ht-termpair+)
%   \\ ht-general&$\leftarrow$&((term$\ast$) (height$\ast$))
%   \\ aux&$\leftarrow$&string~$\mid$~symbol~$\mid$~integer
\\ skel-decl&$\leftarrow$&(one-term-decl term+)
\\ &$\mid$&(two-term-decl (term term)+)
%   \\ &$\mid$&(\sym{neqlist} (term+)+)
%   \\ &$\mid$&(\sym{subsort} subsort+)
%   \\ &$\mid$&(\sym{fn-of} function+)
%   \\ &$\mid$&(\sym{decl} symbol general+)
\\ &$\mid$&(\sym{facts} (id term*)*)
\\ &$\mid$&(\sym{priority} (node integer)+)
%   \\ subsort&$\leftarrow$&(aux term+)
%   \\ function&$\leftarrow$&(aux (term term)+)
%   \\ general&$\leftarrow$&((term$\ast$) (node$\ast$))
\\ node&$\leftarrow$&(integer integer)
  \end{tabular}
\end{center}
\caption[Declaration syntax]{Declaration syntax.  See
  Tables~\ref{tab:basic_term} and~\ref{tab:dh_term} for the algebra
  syntax, which defines the $\scap{term}$ symbols.  See
  Table~\ref{tab:goal_syntax} for formulas \textsc{conclusion}.  }
\label{tab:decl_syntax}
\end{table}

\section{Command-line options}
\label{sec:options}

The following command-line options are defined for the {\cpsa} program:

\begin{itemize}
\index{options!output}
\item `o' (``output'') -- specifies a file to which the output will be written.
  Usage: \texttt{-o FILE} or \texttt{--output=FILE}.
  \index{options!limit} \index{step limit}
\item `l' (``limit'') -- specifies a step limit (see
  Section~\ref{sec:bounds}).  By default, the step limit is 2000.
  Usage: \texttt{-l INT} or \texttt{--limit=INT}.
  \index{options!bound} \index{strand bound}
\item `d' (``depth'') -- specifies a depth limit (see
  Section~\ref{sec:bounds}).  The depth limit is infinite by default.
  Usage: \texttt{-d INT} or
  \texttt{--depth=INT}. \index{options!depth} \index{depth limit}
\item `b' (``bound'') -- specifies a strand bound (see Section~\ref{sec:bounds}).
  By default, the strand bound is 12.
  Usage: \texttt{-b INT} or \texttt{--bound=INT}.
\index{options!margin}
\item `m' (``margin'') -- specifies the number of characters allowed
  per line in the output. By default, the margin is 72.
  Usage: \texttt{-m INT} or \texttt{--margin=INT}.
\index{options!expand}
\item `e' (``expand'') -- expands macros only.  For use in debugging
  macros; when run with this option, the tool does not analyze its
  input.
\index{options!noanalyze}
\item `z' (``noanalyze'') -- loads the input but does not analyze.
    This could be used for checking whether any errors are present in
    the input.
\index{options!noisochk}
\item `n' (``noisochk'') -- disable isomorphism checks.  {\bf Not
  recommended except in rare circumstances.}  When the tool produces a
  new skeleton, it only gets a number in the output if no isomorphic
  skeleton has already been found.  With isomorphism checking
  disabled, many analyses that otherwise terminate will not terminate.
  However, there is one advantage.  \index{seen child} Normally, when
  a skeleton is found more than once in an analysis, its operation
  field (see Section~\ref{sec:kerberos2}) is displayed only for the
  first time that skeleton is found.  Other instances are not
  annotated with an operation field but instead simply link to the
  isomorphic skeleton.
\index{options!check-nonces} \index{check-nonces option}
\item `c' (``check-nonces'') -- check nonces first.  Normally, the
  tool will consider the largest possible critical term when attempting
  to solve an unrealized node.  This option flips the default behavior.
  Some analyses terminate only with this option, while others perform
  better with this option.
\index{options!try-old-strands} \index{try-old-strands option}
\item `t' (``try-old-strands'') -- try old strands first.  Normally,
  when there are unrealized nodes on multiple strands, the tool will
  focus on the highest-numbered (rightmost in the diagram) strand with
  unrealized nodes.  This option flips the default behavior.  Some
  analyses terminate only with this option, while others perform
  better with this option.
\index{options!reverse-nodes} \index{reverse-nodes option}
\item `r' (``reverse-nodes'') -- try younger nodes first.  Normally,
  when a strand containing unrealized nodes is chosen for analysis, if
  there are multiple unrealized nodes, the earliest (oldest) node is
  analyzed.  This option flips the default behavior.  Some analyses
  terminate only with this option, while others perform better with
  this option.
\index{options!goals-sat} \index{goals-sat option}
\item `g' (``goals-sat'') -- stop when goals are satisfied.  Normally,
  a \texttt{defgoal} is converted to a skeleton and analyzed, and only
  shapes are compared to the goals.  With this option enabled, the
  goals are checked against every intermediate skeleton, and branches
  of the analysis in which the goals are satisfied are not further
  pursued.  The output analysis will show the skeletons in which the
  goals were first satisfied, and any shapes in which the goals were
  not satisfied.  In a sense, running {\cpsa} with this option asks
  the question: what are all the ways in which this goal may not be
  satisfied?
\index{options!algebra} \index{algebra} \index{Diffie-Hellman!algebra}
\item `a' (``algebra'') -- sets the algebra to use for the input.  The
  analysis tool expects to run with a single algebra for an entire
  session.  By default the tool uses the \texttt{basic} algebra; an
  input file using the Diffie-Hellman algebra will cause an error
  unless you specify the \texttt{diffie-hellman} algebra via this
  option.
  Usage: \texttt{-a STRING} or \texttt{--algebra=STRING}.
\index{options!show-algebras}
\item `s' (``show-algebras'') -- causes the tool to print a list of the
  allowed algebras, instead of performing any analysis.
\index{options!help}
\item `h' (``help'') -- causes the tool to print a list of
  command-line options, and also to print the directory in which the
  package's documentation exists.  When run with this option, the tool
  will not perform any analysis.
\index{options!version}
\item `v' (``version'') -- causes the tool to simply print its version
  number rather than performing any analysis.

\end{itemize}

Additionally,

\begin{itemize}
  \item To protect {\cpsa} against memory exhaustion, run {\cpsa} with
    the command-line options \verb|+RTS -M512m -RTS|.

  \item To make use use of parallelism on an $n$-processor machine,
    start {\cpsa} with the command-line options \verb|+RTS -N|$n$\verb| -RTS|.
\end{itemize}

See the GHC User's Guide \cite{ghcUsersGuide} for documentation on these and other options.

\subsection{Heralds}
\label{sec:heralds}
\index{herald}
At the top of each {\cpsa} input file is, optionally, a \texttt{herald}
that is an alternate way to specify command-line options.  All of the
examples in the examples directory contain heralds.  The herald, in
addition to setting certain options for the analyzer, is also included
in the output file, which may be useful to the analyst in
distinguishing one analysis from a similar, related one.

Options are included in the herald by constructing an S-expression
with first element equal to the long name of an option, and the second
element (if any) set to be the parameter for that option.  For instance,
\texttt{(algebra diffie-hellman)} is included for all the examples that
use the \texttt{diffie-hellman} algebra.

Command-line options override the herald.  Also, certain command-line
options are not allowed in the herald, specifically, ``output'',
``help'', ``version'', and ``show-algebras''.

%%% Local Variables:
%%% mode: latex
%%% TeX-master: "cpsa4manual"
%%% End:
