\chapter{Introduction}
\label{ch:intro}

{\cpsa}, the Cryptographic Protocol Shapes Analyzer, is a software tool
designed to assist in the design and analysis of cryptographic protocols.
A cryptographic protocol is a specific pattern of interaction between
principals.  TLS and IKE are some examples of well-known cryptographic protocols.

The tool takes as input a protocol definition and a partial
description of an execution, each built within a particular formal
model by the user.  It attempts to produce descriptions of \emph{all}
possible executions of the protocol that complete the partial
description, consistent with the presence of a powerful network adversary
capable of diverting, altering, replaying, or dropping network messages.
Such an adversary may be able to manipulate honest participants into an
unexpected execution, breaking a secrecy or authentication property that
the protocol was intended to achieve.

Naturally, there are infinitely many possible executions of a useful
protocol, since different participants can run it with varying
parameters, and the participants can run it repeatedly.  However, for
many naturally occurring protocols, there are only finitely many of
these runs that are \emph{essentially} different.  Indeed, there are
frequently very few, often just one or two, even in cases where the
protocol is flawed.  We call these minimal, essentially different
executions the \emph{shapes} of the protocol.  Authentication and
secrecy properties are easy to ``read off'' from the shapes, as are
attacks and anomalies.

The analysis performed by {\cpsa} is done within a pure Dolev-Yao
model~\cite{DolevYao83}; as such, the analysis reveals only structural
flaws in the protocols it analyzes.  It cannot detect flaws in underlying
cryptographic algorithms or in actual implementations of protocol participants.

The purpose of this document is to provide the background required to
make effective use of the {\cpsa} software distribution.

\section{Recommended reading}

If you are new to {\cpsa}, it is recommended that you first read
Part~\ref{part:basic}, which is introductory in nature and presented
as a tutorial.  Chapter~\ref{ch:setup} discusses how to download,
install, and run the tool.  Chapter~\ref{ch:basic} begins the tutorial
describing how to use the tool, and Chapter~\ref{ch:algebra} describes
some of the more important additional features.

Readers, especially those without direct access to an experienced user
of the tool, are encouraged to attempt the ``explorations'' present
from Chapter~\ref{ch:basic} onwards.  The purpose of the explorations
is to give the reader experience in the use of the tool and a chance
to test his or her understanding of its features.  After reading
through the first part, you should be ready to attempt to use the tool
to analyze a protocol that interests you, whether it be an existing
protocol or one you need to design.

When you are ready for a deeper base of knowledge, read
Part~\ref{part:intermediate}.  The chapters in
Part~\ref{part:intermediate} will be helpful as you cross from trying
to understand how to use the tool to trying to impact your work on
protocol design through use of the tool.  Chapter~\ref{ch:algorithm}
will be of general use as you try to understand the analyses {\cpsa}
conducts, and Chapter~\ref{ch:declarations} will help you narrow the
tool's focus to what interests you.

Part~\ref{part:advanced} deals with special-purpose features of the tool.
Chapter~\ref {chap:channels:state} deals with stateful protocols and
Chapter~\ref{ch:goals} deals with logic-based protocol goals.  You
should read these chapters if those features seem important to you.

Part~\ref{part:reference} is reference material.
Chapter~\ref{ch:troubleshooting} contains reference material about
dealing with errors of various sorts that arise during the use of
{\cpsa}, while Chapter~\ref{ch:input} documents the complete syntax of
the tool.

\section{Tool components}

The distribution includes a number of separate executable command-line
tools.  Of these, three are key components of the core expected
workflow: \texttt{cpsa4}, \texttt{cpsa4graph}, and \texttt{cpsa4shape}.
The other tools are auxiliary utilities most general users will not
need to use, although \texttt{cpsa4diff} is of some use to a general
user when updating the tool.

The \texttt{cpsa4} program takes as input one or more analysis problems
(at least, a protocol and a partial description of an execution), and
analyzes them one at a time.  It outputs the full, step-by-step
analysis for each problem, ultimately describing all possible full
executions that complete the partial execution.

The \texttt{cpsa4shape} program takes an analysis and reduces it to an
analysis that skips directly from analysis input to the set of shapes
associated with that input.

The \texttt{cpsa4graph} program takes an analysis (from either
\texttt{cpsa4} or \texttt{cpsa4shape}) and formats it into an
\textsc{xhtml} file and includes \textsc{svg}(scalable vector
graphics) diagrams of each partial execution as well as the overall
branching pattern of the analysis.

%The shapes analysis is performed within a pure Dolev-Yao
%model~\cite{DolevYao83}.  {\cpsa}'s search is based on a high-level
%algorithm that was claimed to be complete, i.e.\@ every shape can in
%fact be found in a finite number of
%steps~\cite{DoghmiGuttmanThayer07,Guttman11}.  Further theoretical
%work showed classes of executions that are not found by the algorithm,
%however it also showed that every omitted execution requires an
%unnatural interpretation of a protocol's roles.  Hence the algorithm
%is complete relative to natural role semantics.
%Appendix~\ref{sec:omitted executions} discusses omitted executions.

%\section{Overview}
%
%A {\cpsa} release includes several programs, an analyzer, and various
%tools used to interpret the results.  The analyzer, \texttt{cpsa4},
%provides support for several algebras, one of which is the Basic
%Crypto Algebra.  Programs that assist in the interpretation of results
%are \texttt{cpsa4shape} and \texttt{cpsa4graph}.  The analyzer prints
%the steps it used to solve each problem.  The \texttt{cpsa4shapes}
%program extracts the shapes discovered by an analyzer run.  The
%\texttt{cpsa4graph} program graphs both forms of output using Scalable
%Vector Graphics~(\textsc{svg}).  A standards-compliant browser such as
%FireFox or Safari displays the generated diagrams.

The expected work flow follows.  An analysis problem is entered using
an ordinary text editor, preferably one with support for Lisp syntax.
The \texttt{cpsa4} program uses an S-expression-based syntax for both
input and output.  S-expression is an abbreviation for a Symbolic
Expression (as in the Lisp programming language).

The body of the input consists of two forms: \texttt{defprotocol}
statements that describe a protocol, and \texttt{defskeleton}
statements that describe a partial execution of a protocol.  The exact
details of both forms depend on the message algebra specified by the
protocol.
%Protocols that specify \texttt{basic} as their algebra get
%an implementation of the Basic Crypto Algebra~(\textsc{bca}) described
%in the next section.  A complete grammar for \texttt{cpsa4} input is
%displayed in Table~\ref{tab:syntax} on Page~\pageref{tab:syntax}.

%Problem statement errors in the input are detected by running the
%analyzer.  Many error reports are of the form that allow editors such
%as Emacs to move its cursor to location of the problem.

%There are two classes of problem statement errors: syntax and semantic
%errors.  Correcting syntax errors is straightforward, but correcting
%semantic errors requires an understanding of the core data structures.
%Section~\ref{sec:semantic errors} describes their correction.

%Once the problem statement errors have been eliminated, the analyzer
Assuming there are no errors in the input, the analyzer
will produce output as a text document.  The text document
contains each step used to derive a shape from a problem statement.
It is common to filter the output using the \texttt{cpsa4shapes}
program, and look only at the computed shapes associated with each
problem statement.

The \texttt{cpsa4graph} program is applied to the output to produce a
more readable, hyperlinked \textsc{xhtml} document that can be displayed
in a standards-compliant web browser.  See Chapter~\ref{ch:tools} for
more information on the programs that come with a {\cpsa} distribution.

% The guide is no longer maintained.
%
%% The {\cpsa} User Guide contains
%% the up-to-date description of \texttt{cpsa4graph} generated documents.
%% The guide is also the place to find command-line usage information for
%% all programs in a release.  The user guide is an \textsc{xhtml}
%% document delivered with the software.

%The input may optionally start with a \texttt{herald} form.  The form
%contains a title for the run and an association list.  The association
%list allows options normally specified on the command line to be
%specified within an input file.  In the following example, the herald
%form specifies a strand bound of 12 in a way that is equivalent to the
%command line option \texttt{--bound=12}.

%\begin{quote}
%\begin{verbatim}
%(herald Needham-Schroeder (bound 12))
%\end{verbatim}
%\end{quote}

%%% Local Variables:
%%% mode: latex
%%% TeX-master: "cpsa4manual"
%%% End:
